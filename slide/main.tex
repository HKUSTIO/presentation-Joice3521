\documentclass[aspectratio=169]{beamer}  % 16:9 aspect ratio

% Use a clean theme as base
\usetheme{default}
\usecolortheme{default}

% Custom colors from HKUST logo
\definecolor{hkustblue}{RGB}{0, 51, 119}    % Navy blue from logo
\definecolor{hkustgold}{RGB}{180, 141, 61}  % Golden brown from logo
\definecolor{lightgray}{RGB}{236, 240, 241}

% Customize the appearance
\setbeamercolor{structure}{fg=hkustblue}
\setbeamercolor{background canvas}{bg=white}
\setbeamercolor{normal text}{fg=hkustblue}
\setbeamercolor{frametitle}{fg=hkustblue,bg=white}
\setbeamercolor{itemize item}{fg=hkustgold}
\setbeamercolor{itemize subitem}{fg=hkustgold}
\setbeamercolor{block title}{fg=white,bg=hkustblue}
\setbeamercolor{block body}{fg=hkustblue,bg=lightgray}
\setbeamercolor{title}{fg=hkustblue}
\setbeamercolor{subtitle}{fg=hkustgold}

% Remove navigation symbols
\setbeamertemplate{navigation symbols}{}
\setbeamertemplate{footline}[frame number]
\setbeamerfont{footline}{size=\scriptsize}

\renewcommand{\raggedright}{\leftskip=0pt \rightskip=12pt plus 0cm}


% Customize frame title
\setbeamertemplate{frametitle}{
    \vspace*{0.2cm}
    \insertframetitle
    \vspace*{0.2cm}
    \begin{beamercolorbox}[wd=\paperwidth,ht=0.2pt]{structure}
    \end{beamercolorbox}
}

% Customize itemize bullets
\setbeamertemplate{itemize item}{\small\raise0.5pt\hbox{\textbullet}}
\setbeamertemplate{itemize subitem}{\tiny\raise1.5pt\hbox{\textbullet}}

% Packages
\usepackage{graphicx}
\usepackage{amsmath}
\usepackage{amsfonts}
\usepackage{hyperref}

% Title page information
\title{Chapter 9 Structural Empirical Analysis of Vertical Contracting}
\subtitle{Theory}
\author{Presented by Yangfan Sun}
\institute{Hong Kong University of Science and Technology}
\date{March 2025}

\begin{document}

% Title page
\begin{frame}[plain]
    \titlepage
\end{frame}

% Table of contents
\begin{frame}{Outline}
    \tableofcontents
\end{frame}

% Section 1
\section{Introduction}
\begin{frame}{Introduction}
    \begin{itemize}
        \item One key feature of many industries is a vertical supply chain characterized by an oligopolistic market structure at each level of the chain. \vspace{.2cm}
        \item A model that allows for such margins of adjustment will often be necessary for an accurate prediction of the effects of a policy. \vspace{.2cm}
        \item From theoretical to empirical work, \vspace{.2cm}
        \begin{itemize}
            \item Must work with underlying theoretical models that feature both contracting and competition, yet are tractable and estimable.\vspace{.2cm}
            \item Requires more detailed data and rich institutional knowledge. \vspace{.2cm}
        \end{itemize}
    \end{itemize}
\end{frame}



% Section 2
\section{Basics}
\begin{frame}{Basics}
    \begin{itemize}
        \item Setting \vspace{.2cm}
        \begin{itemize}
            \item A upstream seller $U$ and a downstream buyer $D$ agree on a contract $\mathcal{C}$ from some feasible set. \vspace{.2cm}
            \item $\mathcal{C} = \{y, t\}$, $y$ includes other provisions and $t$ is a lump-sum transfer. \vspace{.2cm}
            \item Firms' payoffs: $\Pi_U(\mathcal{C})\equiv \pi_U(y) + t$; $\Pi_D(\mathcal{C})\equiv \pi_D(y) - t$ \vspace{.2cm}
        \end{itemize}
        \item The maximization problem \vspace{.2cm}
        \begin{equation}
            \begin{aligned}
            & \max_{(y, t) \in \mathcal{Y} \times \mathbb{R}} \quad \pi_U(y) + t \\
            & \text{s.t.} \quad \pi_D(y) - t \geq \overline{\Pi}_D
            \end{aligned}
            \end{equation}
    \end{itemize}
\end{frame}



\begin{frame}{Basics - example: successive monopoly setting}
    \begin{itemize}
        \item A monopolist manufacturer sells a product to a monopolist retailer. \vspace{.2cm}
        \begin{itemize}
            \item Vertically integrated: $p^m(c)\equiv \arg\max (p-c_m-c_R)D(p)$, $p^m(c_M+c_R)$ maximizes the bilateral surplus. \vspace{.2cm}
            \item Price unilaterally: $p^m(w+c_R)$ maximizes the retailer's profit. \vspace{.2cm}
            \item $p^m(w+c_R) > p^m(c_M+c_R)$: double marginalization problem. \vspace{.2cm} 
        \end{itemize} 
    \end{itemize}
\end{frame}



\begin{frame}{Basics - example: negotiation by Nash bargaining}
    \begin{itemize}
        \item The parties will agree to a contract $\mathcal{C} = \{y, t\}$ that solves 
        \begin{align}
            \max_{C \in \mathcal{C}^+} \left[ \pi_D(y) - \overline{\Pi}_D \right]^b \cdot \left[ \pi_U(y) - \overline{\Pi}_U \right]^{1-b}
        \end{align}
        \item Take derivatives with respect to $y$ and $t$, and we have
        \begin{align}
            \frac{\partial \pi_D(y)}{\partial y_k} + \frac{\partial \pi_U(y)}{\partial y_k} = 0 \quad \text{for } k = 1, \ldots, K
        \end{align}
            
    \end{itemize}
\end{frame}



\begin{frame}{Basics - example: negotiation by Nash bargaining (continued)}
    \begin{itemize}
        \item Consider the contract $\mathcal{C}$ only includes the wholesale price $w$. \vspace{.2cm}
        \item We will have
        \begin{align}
            (w - c_M) \overline{D}'(w) + \overline{D}(w) = ( \frac{b}{1 - b} ) (\frac{w-c_M}{p^m ( w + c_R) - (w + c_R)}) \overline{D}(w)
        \end{align}
        where $\overline{D}(w) \equiv D(p^m(w+c_R))$ is the retail demand conditioning on the wholesale price $w$. \vspace{.2cm}
        \begin{itemize}
            \item $b\rightarrow0$: successive monopoly setting. \vspace{.2cm}
            \item $b\rightarrow1$: implies $w\rightarrow c_M$, maximizing bilateral surplus. \vspace{.2cm}
        \end{itemize}
    \end{itemize}

\end{frame}



% Section 3
% Subsection 1
% \section{Multilateral settings with externalities}
\section{Non-cooperative bargaining models}
\begin{frame}{Multilateral settings with externalities}
    \begin{itemize}
        \item Previous case: one-to-one vertical contracting. \vspace{.2cm}
        \item Things can be complicated. \vspace{.2cm}
        \begin{itemize}
            \item One-to-many contracting: non-cooperative bargaining \vspace{.2cm}
            \item Many-to-many contracting: Nash-in-Nash bargaining \vspace{.2cm}
        \end{itemize}
        \item Non-cooperative bargaining: offer \& bidding games \vspace{.2cm}
    \end{itemize}
\end{frame}



\subsection{The offer game}
\begin{frame}{The offer game - introduction}
    \begin{itemize}
        \item Contracting parties make take-it-or-leave-it offers. Lump-sum transfers are feasible. \vspace{.2cm}
        \item Whether the offer can be known by the other party. \vspace{.2cm}
        \begin{itemize}
            \item Public offers \vspace{.2cm}
            \item Private offers \vspace{.2cm}
        \end{itemize}
    \end{itemize}
\end{frame}



\begin{frame}{The offer game - introduction (continued)}
    \begin{itemize}
        \item The principal can sign a bilateral contract $\mathcal{C}_j=(q_j,t_j)$ with agent $j=1,...,J$. \vspace{.2cm}
        \item The principal's payoff is $\pi_P(\mathbf{q})+\sum_{j}t_j$ and agent $j$'s payoff is $\pi_j(\mathbf{q})-t_j$. \vspace{.2cm}
        \item Condition $W$: the joint payoff depends only on the aggregate trade $Q\equiv\sum_jq_j$. When it holds, we also assume that all efficient trade profiles $\mathbf{q}^*\in\mathcal{Q}^*$ have the same aggregate trade $Q^*$. \vspace{.2cm}
        \begin{align*}
            \mathcal{Q}^* \equiv \arg \max_{\mathbf{q} \in \mathbb{R}^J} \pi_P(\mathbf{q}) + \sum_j \pi_j(\mathbf{q})
        \end{align*}
    \end{itemize}
\end{frame}



\begin{frame}{The offer game - public offers}
    \begin{itemize}
        \item Focus on the equilibria where all agents accept the contract. \vspace{.2cm}
        \item Agent $j$ will accept the contract if and only if
        \begin{align*}
            \pi_j(\mathbf{q}) - t_j \geq \pi_j(0, \mathbf{q}_{-j})
        \end{align*}
        \item Given this, the principal will offer $\hat{\mathbf{q}}$ solving
        \begin{align}
            \max_{\mathbf{q} \in \mathcal{R}^J} \{\pi_P(\mathbf{q}) + \sum_{j=1} \pi_j(\mathbf{q})\}  - \sum_{j=1} \pi_j(0, \mathbf{q}_{-j})
        \end{align}
        \item The inefficiency comes from the externalities on non-traders. If there are externalities, the principal has the incentive to distort to lower the reservation payoffs. \vspace{.2cm}
    \end{itemize}
\end{frame}



\begin{frame}{The offer game - public offers (continued)}
    \begin{block}{Proposition 1}
        \textit{In the public-offer game with lump-sum transfers and absent externalities on non-traders, the equilibrium trade profile $\hat{\mathbf{q}}$ is efficient, i.e., $\hat{\mathbf{q}}\in \mathcal{Q}^*$.}
    \end{block}\vspace{.2cm}
    \begin{block}{Proposition 2}
        \textit{Assume Condition $W$ holds and suppose that the aggregate trade
        in an equilibrium trade profile of the public-offer game is $\hat{Q}$. Then with positive (or negative) externalities on non-traders, $Q\leq$ (or $\geq$) $Q^*$.}
    \end{block}
\end{frame}



\begin{frame}{The offer game - public offers (continued)}
    \textit{Proof of Proposition 2} \vspace{.2cm}
    \begin{itemize}
        \item Suppose that externalities on non-traders are positive. \vspace{.2cm}
        \item The minimized value of reservation utility becomes \vspace{.2cm}
        \begin{equation*}
            \begin{aligned}
                R(Q) \equiv & \min_{q \in \mathbb{R}^J} \sum_j \pi_j(0, \mathbf{q}_{-j}) \\
                & \text{s.t. } \sum_j q_j = Q
            \end{aligned}
        \end{equation*}
        Note that $R(\cdot)$ is a non-decreasing function. \vspace{.2cm}
    \end{itemize}
\end{frame}



\begin{frame}{The offer game - public offers (continued)}
    \textit{Proof of Proposition 2 (cont'd)} \vspace{.2cm}
    \begin{itemize}
        \item The principal's problem becomes \vspace{.2cm}
        \begin{align*}
            \max_{Q} & \quad \Pi(Q) - R(Q)
        \end{align*}
        \item Suppose that $\hat{Q}>Q^*$. By definition of $R(\cdot)$, we have $\Pi(\hat{Q}) - \Pi(Q^*) > R(\hat{Q}) - R(Q^*) \geq 0$, which contradicts the optimality condition as $\hat{Q}$ should be chosen instead of $Q^*$. So we must have $\hat{Q}\leq Q^*$.\hfill$\square$ \vspace{.2cm}
    \end{itemize}
\end{frame}



\begin{frame}{The offer game - public offers (continued)}
    \begin{itemize}
        \item Negative externalities will lead to large trade, while positive externalities will lead to small trade. \vspace{.2cm}
        \begin{itemize}
            \item Exclusive contract: hinder potential entrants and reduce potential benefits to non-traders. \vspace{.2cm}
            \item M\&A contract: reduce competition and increase the benefits to other competing firms. \vspace{.2cm}           
        \end{itemize}
    \end{itemize}

\end{frame}



\begin{frame}{The offer game - private offers}
    \begin{itemize}
        \item The offer can only be observed by the agent. \vspace{.2cm}
        \item Assume agents hold passive beliefs: they believe other agents received their equilibrium offers even when they receive an unexpected offer. \vspace{.2cm}
        \item Similar to the public-offer game, the equlibrium trade profile $\hat{\mathbf{q}}=\{\hat{q_1},...,\hat{q_J}\}$ 
        \begin{align*}
            \hat{\mathbf{q}} \in \arg \max_{q \in \mathbb{R}^J} \pi_P(\mathbf{q}) + \sum_j [ \pi_j(q_j, \hat{\mathbf{q}}_{-j}) - \pi_j(0, \hat{\mathbf{q}}_{-j})]
        \end{align*}
        \item The inefficiency comes from the externality on efficient traders.
    \end{itemize}
\end{frame}



\begin{frame}{The offer game - private offers (continued)}
    \begin{block}{Proposition 3}
        \textit{In the private-offer game with lump-sum transfers:\\
        (i) If there are no externalities on efficient traders, then any passive beliefs equilibrium trade profile is efficient.}\\
        \textit{(ii) Assume Condition $W$ holds and and let $\hat{Q}$ be the aggregate trade in a passive beliefs equilibrium. If externalities on efficient traders are positive (or negative), then $\hat{Q} \leq$ (or $\geq$) $Q^*$.}
    \end{block}
\end{frame}



\begin{frame}{The offer game - private offers (continued)}
    \textit{Proof of Proposition 3 (i)} \vspace{.2cm}
    \begin{itemize}
        \item Notice that for any passive beliefs equilibrium trade profile $\hat{q}$, and any efficient trade profile $q^*\in Q^*$, we have
        \begin{equation}
            \begin{aligned}
                \pi_P(\hat{\mathbf{q}}) + \sum_j \pi_j(\hat{q}_j,\hat{\mathbf{q}}_{-j}) & \geq \pi_P(\mathbf{q}^*) + \sum_j \pi_j(q_j^*,\hat{\mathbf{q}}_{-j})\\
                & = \pi_P(\mathbf{q}^*) + \sum_j \pi_j(q_j^*, \mathbf{q}^*_{-j})
            \end{aligned}    
        \end{equation}
        Together they imply $\hat{\mathbf{q}}$ is efficient. \vspace{.2cm}
    \end{itemize}
\end{frame}



\begin{frame}{The offer game - private offers (continued)}
    \textit{Proof of Proposition 3 (ii)} \vspace{.2cm}
    \begin{itemize}
        \item Suppose there are negative externalities on efficient traders but $\hat{Q}<Q^*$. \vspace{.2cm}
        \item Under Condition $W$, there is some efficient trade profile $q^*$ such that $\sum_j q^*_j = Q^*$ and $\hat{q}_j<q_j^*$ for all $j$. \vspace{.2cm}
        \begin{equation*}
            \begin{aligned}
                \pi_P(\hat{\mathbf{q}}) + \sum_j \pi_j(\hat{q}_j, \hat{\mathbf{q}}_{-j}) & \geq \pi_P(\mathbf{q}^*) + \sum_j \pi_j(q_j^*, \hat{\mathbf{q}}_{-j})  \\
                & > \pi_P(\mathbf{q}^*) + \sum_j \pi_j(q_j^*, \hat{\mathbf{q}}^*_{-j})
            \end{aligned}
        \end{equation*}
        which contradicts $\mathbf{q}^*$ being efficient. Hence, we must have $\hat{Q}\geq Q^*$.\hfill$\square$
    \end{itemize}
\end{frame}



\subsection{The bidding game}
\begin{frame}{The bidding game}
    \begin{itemize}
        \item Multiple principal make offers to the single agent, who then decide whether to accept or reject each offer. \vspace{.2cm}
        \item Only unilateral contract deviations are possible. \vspace{.2cm}
        \item It is possible for deviating contract offer to induce the agent to reject the offer from a rival principal. \vspace{.2cm}
    \end{itemize}
\end{frame}



\begin{frame}{The bidding game (continued)}
    An example \vspace{.2cm}
    \begin{itemize}
        \item There are two manufacturers. Each manufacturer $j$ must earn her marginal contribution to the joint monopoly profit given the trade with the other manufacturer.
        \begin{align}
            t_j - c_j q_j^* = [ P(q_1^* + q_2^*)(q_1^* + q_2^*) - c_1 q_1^* - c_2 q_2^*] - [ P(q_j^*, 0)q_j^* - c_j q_j^*]
        \end{align}
        \item Have the incentive to provide a exclusive offer.
        \begin{align}
            q_k^e = \arg \max_{q_k} P(q_k, 0) q_k - c_k q_k
        \end{align}
    \end{itemize}
\end{frame}



\begin{frame}{The bidding game (continued)}
    \begin{itemize}
        \item Less is known about equilibrium outcomes in settings with contracting externalities for bidding games than for offer games. \vspace{.2cm}
        
    \end{itemize}
\end{frame}



% Subsection 2
\section{Nash-in-Nash bargaining}
\begin{frame}{Nash-in-Nash bargaining}
    \begin{itemize}
        \item Consider a setting with $I$ sellers and $J$ buyers. \vspace{.2cm}
        \item Each pair $ij$ may agree to a contract $\mathbb{C}_{ij}\in\mathcal{C}_{ij}$. Given a collection of contracts between all pairs $i$ and $j$, $\mathbb{C}\equiv\{\mathbb{C}_{ij}\}$, downstream firm $j$'s payoff is $\Pi_{Dj}(\mathbb{C})$ and upstream firm $i$'s payoff is $\Pi_{Uj}(\mathbb{C})$. \vspace{.2cm}
        \item Contracts $\hat{\mathbb{C}}\equiv \{\hat{\mathbb{C}}_{ij}\}$ constitute a Nash-in-Nash equilibrium if for all $ij$ such that $\hat{\mathbb{C}}_{ij}\neq \mathbb{C}$,  \vspace{.2cm}
        \begin{equation}
        \begin{aligned}
            \hat{\mathbb{C}}_{ij} \in \arg &\max_{\mathbb{C}_{ij} \in \mathcal{C}_{ij}^+ (\hat{\mathbb{C}}_{-ij})} [ \Pi_{Dj}(\mathbb{C}_{ij},\hat{\mathbb{C}}_{-ij}) - \Pi_{Dj}(\mathbb{C}_{0},\hat{\mathbb{C}}_{-ij})]^{b_{ij}}\\
            & \times [ \Pi_{Uj}(\mathbb{C}_{ij},\hat{\mathbb{C}}_{-ij}) - \Pi_{Uj}(\mathbb{C}_{0},\hat{\mathbb{C}}_{-ij})]^{1-b_{ij}}
        \end{aligned}
        \end{equation}
    \end{itemize}
\end{frame}



\begin{frame}{Nash-in-Nash bargaining (continued)}
    \begin{itemize}
        \item A collection of contracts is a Nash-in-Nash equilibrium if each pair's contract solves the bilateral Nash bargaining problem \textit{taking the contracts agreed by all other pairs as given.} \vspace{.2cm}
        \item However, Nash-in-Nash equilibria may involve unreasonable payoff predictions. \vspace{.2cm}
    \end{itemize}
\end{frame}



% Section 4
\begin{frame}{Nash-in-Nash bargaining (continued)}
    An example \vspace{.2cm}
    \begin{itemize}
        \item There are two manufacturers and one retailer, with equal bargaining power between them. Let $Q^m$ denote the joint monopoly sales level for the vertical structure. \vspace{.2cm}
        \item One possible Nash-in-Nash equilibrium is $q_1 = Q^m$, $q_2 = 0$. Manufacturer 1 and the retailer share the profits equally. \vspace{.2cm}
        \item However, in the bidding/offer game, all of the profits would be earned by the retailer. \vspace{.2cm}
    \end{itemize}
\end{frame}



\begin{frame}{Nash-in-Nash bargaining (continued)}
    Nash-in-Nash with Threat of Replacement (Ho and Lee, 2019) \vspace{.2cm}
    \begin{itemize}
        \item The retailer can credibly threaten to replace the manufacturer with a new manufacturer. \vspace{.2cm} 
        \item Under the NNTR protocol, the equilibrium becomes $q_1 = Q^m$, but the retailer earns all of the profits. \vspace{.2cm}
    \end{itemize}
\end{frame}



\end{document}