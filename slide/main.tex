\documentclass[aspectratio=169]{beamer}  % 16:9 aspect ratio

% Use a clean theme as base
\usetheme{default}
\usecolortheme{default}

% Custom colors from HKUST logo
\definecolor{hkustblue}{RGB}{0, 51, 119}    % Navy blue from logo
\definecolor{hkustgold}{RGB}{180, 141, 61}  % Golden brown from logo
\definecolor{lightgray}{RGB}{236, 240, 241}

% Customize the appearance
\setbeamercolor{structure}{fg=hkustblue}
\setbeamercolor{background canvas}{bg=white}
\setbeamercolor{normal text}{fg=hkustblue}
\setbeamercolor{frametitle}{fg=hkustblue,bg=white}
\setbeamercolor{itemize item}{fg=hkustgold}
\setbeamercolor{itemize subitem}{fg=hkustgold}
\setbeamercolor{block title}{fg=white,bg=hkustblue}
\setbeamercolor{block body}{fg=hkustblue,bg=lightgray}
\setbeamercolor{title}{fg=hkustblue}
\setbeamercolor{subtitle}{fg=hkustgold}

% Remove navigation symbols
\setbeamertemplate{navigation symbols}{}
\setbeamertemplate{footline}[frame number]
\setbeamerfont{footline}{size=\scriptsize}


% Customize frame title
\setbeamertemplate{frametitle}{
    \vspace*{0.2cm}
    \insertframetitle
    \vspace*{0.2cm}
    \begin{beamercolorbox}[wd=\paperwidth,ht=0.2pt]{structure}
    \end{beamercolorbox}
}

% Customize itemize bullets
\setbeamertemplate{itemize item}{\small\raise0.5pt\hbox{\textbullet}}
\setbeamertemplate{itemize subitem}{\tiny\raise1.5pt\hbox{\textbullet}}

% Packages
\usepackage{graphicx}
\usepackage{amsmath}
\usepackage{hyperref}

% Title page information
\title{Chapter 9 Structural Empirical Analysis of Vertical Contracting}
\author{Presented by Siyuan (Bruce) Jin and Yangfan Sun}
\institute{Hong Kong University of Science and Technology}
\date{March 2025}

\begin{document}

% Title page
\begin{frame}[plain]
    \titlepage
\end{frame}

% Table of contents
\begin{frame}{Outline - Theory}
    \tableofcontents
\end{frame}

% Section 1
\section{Introduction}
\begin{frame}{Introduction}
    \begin{itemize}
        \item 
    \end{itemize}
\end{frame}

% Section 2
\section{Basics}
\begin{frame}{Basics}
    \begin{itemize}
        \item Setting \vspace{.2cm}
        \begin{itemize}
            \item A upstream seller $U$ and a downstream buyer $D$ agree on a contract $\mathcal{C}$ from some feasible set. \vspace{.2cm}
            \item The contract $\mathcal{C} = \{y, t\}$, $y$ includes other provisions and $t$ is a lump-sum transfer. \vspace{.2cm}
            \item Firms' payoffs: $\Pi_U(\mathcal{C})\equiv \pi_U(y) + t$; $\Pi_D(\mathcal{C})\equiv \pi_D(y) - t$ \vspace{.2cm}
        \end{itemize}
        \item The maximization problem \vspace{.2cm}
        \begin{equation}
            \begin{aligned}
            & \max_{(y, t) \in \mathcal{Y} \times \mathbb{R}} \quad \pi_U(y) + t \\
            & \text{s.t.} \quad \pi_D(y) - t \geq \overline{\Pi}_D
            \end{aligned}
            \end{equation}
    \end{itemize}
\end{frame}

\begin{frame}{Basics - example: successive monopoly setting}
    \begin{itemize}
        \item A monopolist manufacturer sells a product to a monopolist retailer. \vspace{.2cm}
        \begin{itemize}
            \item Vertically integrated: $p^m(c)\equiv argmax (p-c_m-c_R)D(p)$, $p^m(c_M+c_R)$ maximizes the bilateral surplus. \vspace{.2cm}
            \item Price unilaterally: $p^m(w+c_R)$ maximizes the retailer's profit. \vspace{.2cm}
            \item $p^m(w+c_R) > p^m(c_M+c_R)$: double marginalization problem. \vspace{.2cm} 
        \end{itemize} 
    \end{itemize}
\end{frame}

\begin{frame}{Basics - example: negotiation by Nash bargaining}
    \begin{itemize}
        \item The parties will agree to a contract $\mathcal{C} = \{y, t\}$ that solves 
        \begin{align}
            \max_{C \in \mathcal{C}^+} \left[ \pi_D(y) - \overline{\Pi}_D \right]^b \cdot \left[ \pi_U(y) - \overline{\Pi}_U \right]^{1-b}
        \end{align}
        \item Take derivatives with respect to $y$ and $t$, and we have
        \begin{align}
            \frac{\partial \pi_D(y)}{\partial y_k} + \frac{\partial \pi_U(y)}{\partial y_k} = 0 \quad \text{for } k = 1, \ldots, K
        \end{align}
            
    \end{itemize}
\end{frame}


\begin{frame}{Basics - example: negotiation by Nash bargaining (continued)}
    \begin{itemize}
        \item Consider the contract $\mathcal{C}$ only includes the wholesale price $w$. \vspace{.2cm}
        \item We will have
        \begin{align}
            (w - c_M) \overline{D}(w) + \overline{D}(w) = \left( \frac{b}{1 - b} \right) \left( p^* \left( w + c_R \right) - (w - c_R) \right) \overline{D}(w)
        \end{align}
        where $\overline{D}(w) \equiv D(p^m(w+c_R))$ is the retail demand conditioning on the wholesale price $w$. \vspace{.2cm}
        \begin{itemize}
            \item $b\rightarrow0$: successive monopoly setting. \vspace{.2cm}
            \item $b\rightarrow1$: implies $w\rightarrow c_M$, maximizing bilateral surplus. \vspace{.2cm}
        \end{itemize}
    \end{itemize}

\end{frame}

% Section 3
% Subsection 1
% \section{Multilateral settings with externalities}
\section{Non-cooperative bargaining models}
\begin{frame}{Multilateral settings with externalities}
    \begin{itemize}
        \item Previous case: one-to-one vertical contracting. \vspace{.2cm}
        \item Things can be complicated: many-to-many with contracting externalities. \vspace{.2cm}
        \item Two different approaches to analyze more complicated settings. \vspace{.2cm}
        \begin{itemize}
            \item Non-cooperative bargaining \vspace{.2cm}
            \item Nash-in-Nash bargaining \vspace{.2cm}
        \end{itemize}
    \end{itemize}

\end{frame}

\subsection{The offer game}
\begin{frame}{The offer game - introduction}
    \begin{itemize}
        \item Two issues\vspace{.2cm}
        \begin{itemize}
            \item Contracting parties can make take-it-or-leave-it offers or go back and forth. \vspace{.2cm}
            \item Whether the offer can be known by the other party. \vspace{.2cm}
        \end{itemize}
        \item Begin with non-coorperative approach: only one side of the market has multiple parties, and lump-sum transfers are feasible. \vspace{.2cm}
        \begin{itemize}
            \item Public offers \vspace{.2cm}
            \item Private offers \vspace{.2cm}
        \end{itemize}
    \end{itemize}
\end{frame}


\begin{frame}{The offer game - public offers}
    \begin{itemize}
        \item The principal can sign a bilateral contract $\mathcal{C}_j=(q_j,t_j)$ with agent $j=1,...,J$. \vspace{.2cm}
        \item Focus on the equilibria where all agents accept the contract. \vspace{.2cm}
        \item Agent $j$ will accept the contract if and only if
        \begin{align*}
            \pi_j(\mathbf{q}) - t_j \geq \pi_j(0, \mathbf{q}_{-j})
        \end{align*}
        \item Given this, the principal will offer $\hat{\mathbf{q}}$ solving
        \begin{align}
            \max_{\mathbf{q} \in \mathcal{R}^J} \{\pi_P(\mathbf{q}) + \sum_{j=1} \pi_j(\mathbf{q})\}  - \sum_{j=1} \pi_j(0, \mathbf{q}_{-j})
        \end{align}
        where $\sum_{j=1} \pi_j(0, \mathbf{q}_{-j})$ is the reservation payoffs of agents. \vspace{.2cm}
    \end{itemize}
\end{frame}



\begin{frame}{The offer game - public offers}
    \begin{itemize}
        \item When no externalities on non-traders, the outcome is efficient. However, if there are externalities, the principal has the incentive to distort to lower the reservation payoffs. \vspace{.2cm}
        \item Negative externalities will lead to large trade, while positive externalities will lead to small trade. \vspace{.2cm}
        \begin{itemize}
            \item M\&A contract: reduce competition and increase the benefits to other competing firms. \vspace{.2cm}           
            \item Exclusive contract: hinder potential entrants and reduce potential benefits to non-traders. \vspace{.2cm}
        \end{itemize}
        %\item The principal would determine the aggregate trade $Q$, then allocate $Q$ among agents to minimize the reservation payoffs. Then prinpial's problem becomes \vspace{.2cm}
        %\begin{align*}
        %    \max_{Q} \Pi(Q) - R(Q)
        %\end{align*}
        %where $R(Q)$ is a non-increasing function under nagative externalities. \vspace{.2cm}
        %\begin{align*}
        %    R(Q) \equiv \min_{q \in \mathbb{R}^J} \sum_j \pi_j(0, q_{-j})
        %\end{align*}
        %\begin{align*}
        %    \text{s.t.} \quad \sum_j q_j = Q.
        %\end{align*}
    \end{itemize}

\end{frame}


\begin{frame}{The offer game - private offers}
    \begin{itemize}
        \item 
    \end{itemize}


\end{frame}



\subsection{The bidding game}
\begin{frame}{The bidding game}

\end{frame}

% Subsection 2
\section{Nash-in-Nash bargaining}
\begin{frame}{Nash-in-Nash bargaining}
    
\end{frame}





% Section 4
\begin{frame}{Conclusion}

\end{frame}

\end{document} 

    %\begin{block}{Important Block}
    %    Key information goes here
    %\end{block}
